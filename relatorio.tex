%% abtex2-modelo-artigo.tex, v-1.9.5 laurocesar
%% Copyright 2012-2015 by abnTeX2 group at http://www.abntex.net.br/ 
%%
%% This work may be distributed and/or modified under the
%% conditions of the LaTeX Project Public License, either version 1.3
%% of this license or (at your option) any later version.
%% The latest version of this license is in
%%   http://www.latex-project.org/lppl.txt
%% and version 1.3 or later is part of all distributions of LaTeX
%% version 2005/12/01 or later.
%%
%% This work has the LPPL maintenance status `maintained'.
%% 
%% The Current Maintainer of this work is the abnTeX2 team, led
%% by Lauro César Araujo. Further information are available on 
%% http://www.abntex.net.br/
%%
%% This work consists of the files abntex2-modelo-artigo.tex and
%% abntex2-modelo-references.bib
%%

% ------------------------------------------------------------------------
% ------------------------------------------------------------------------
% abnTeX2: Modelo de Artigo Acadêmico em conformidade com
% ABNT NBR 6022:2003: Informação e documentação - Artigo em publicação 
% periódica científica impressa - Apresentação
% ------------------------------------------------------------------------
% ------------------------------------------------------------------------

\documentclass[
	% -- opções da classe memoir --
	article,			% indica que é um artigo acadêmico
	11pt,				% tamanho da fonte
	oneside,			% para impressão apenas no verso. Oposto a twoside
	a4paper,			% tamanho do papel. 
	% -- opções da classe abntex2 --
	%chapter=TITLE,		% títulos de capítulos convertidos em letras maiúsculas
	%section=TITLE,		% títulos de seções convertidos em letras maiúsculas
	%subsection=TITLE,	% títulos de subseções convertidos em letras maiúsculas
	%subsubsection=TITLE % títulos de subsubseções convertidos em letras maiúsculas
	% -- opções do pacote babel --
	english,			% idioma adicional para hifenização
	brazil,				% o último idioma é o principal do documento
	sumario=tradicional
	]{abntex2}


% ---
% PACOTES
% ---

% ---
% Pacotes fundamentais 
% ---
\usepackage{lmodern}			% Usa a fonte Latin Modern
\usepackage[T1]{fontenc}		% Selecao de codigos de fonte.
\usepackage[utf8]{inputenc}		% Codificacao do documento (conversão automática dos acentos)
\usepackage{indentfirst}		% Indenta o primeiro parágrafo de cada seção.
\usepackage{nomencl} 			% Lista de simbolos
\usepackage{color}				% Controle das cores
\usepackage{graphicx}			% Inclusão de gráficos
\usepackage{microtype} 			% para melhorias de justificação
% ---
		
% ---
% Pacotes adicionais, usados apenas no âmbito do Modelo Canônico do abnteX2
% ---
\usepackage{lipsum}				% para geração de dummy text
% ---
		
% ---
% Pacotes de citações
% ---
\usepackage[brazilian,hyperpageref]{backref}	 % Paginas com as citações na bibl
\usepackage[alf]{abntex2cite}	% Citações padrão ABNT
% ---

% ---
% Configurações do pacote backref
% Usado sem a opção hyperpageref de backref
\renewcommand{\backrefpagesname}{Citado na(s) página(s):~}
% Texto padrão antes do número das páginas
\renewcommand{\backref}{}
% Define os textos da citação
\renewcommand*{\backrefalt}[4]{
	\ifcase #1 %
		Nenhuma citação no texto.%
	\or
		Citado na página #2.%
	\else
		Citado #1 vezes nas páginas #2.%
	\fi}%
% ---

% ---
% Informações de dados para CAPA e FOLHA DE ROSTO
% ---
\titulo{Método das Diferenças Finitas Aplicado a Problemas Bidimensionais}
\autor{Jeferson de Oliveira Batista}
\local{Vitória, Brasil}
\data{2016}
% ---

% ---
% Configurações de aparência do PDF final

% alterando o aspecto da cor azul
\definecolor{blue}{RGB}{41,5,195}

% informações do PDF
\makeatletter
\hypersetup{
     	%pagebackref=true,
		pdftitle={\@title}, 
		pdfauthor={\@author},
    	pdfsubject={Método das Diferenças Finitas Aplicado a Problemas Bidimensionais},
	    pdfcreator={Fernando},
		pdfkeywords={valor de contorno}{bidimensionais}{solução}{problemas}{matriz esparsa}, 
		colorlinks=true,       		% false: boxed links; true: colored links
    	linkcolor=blue,          	% color of internal links
    	citecolor=blue,        		% color of links to bibliography
    	filecolor=magenta,      		% color of file links
		urlcolor=blue,
		bookmarksdepth=4
}
\makeatother
% --- 

% ---
% compila o indice
% ---
\makeindex
% ---

% ---
% Altera as margens padrões
% ---
\setlrmarginsandblock{3cm}{3cm}{*}
\setulmarginsandblock{3cm}{3cm}{*}
\checkandfixthelayout
% ---

% --- 
% Espaçamentos entre linhas e parágrafos 
% --- 

% O tamanho do parágrafo é dado por:
\setlength{\parindent}{1.3cm}

% Controle do espaçamento entre um parágrafo e outro:
\setlength{\parskip}{0.2cm}  % tente também \onelineskip

% Espaçamento simples
\SingleSpacing

% ----
% Início do documento
% ----
\begin{document}

% Seleciona o idioma do documento (conforme pacotes do babel)
%\selectlanguage{english}
\selectlanguage{brazil}

% Retira espaço extra obsoleto entre as frases.
\frenchspacing 

% ----------------------------------------------------------
% ELEMENTOS PRÉ-TEXTUAIS
% ----------------------------------------------------------

%---
%
% Se desejar escrever o artigo em duas colunas, descomente a linha abaixo
% e a linha com o texto ``FIM DE ARTIGO EM DUAS COLUNAS''.
% \twocolumn[    		% INICIO DE ARTIGO EM DUAS COLUNAS
%
%---
% página de titulo
\maketitle

% resumo em português


% ]  				% FIM DE ARTIGO EM DUAS COLUNAS
% ---

% ----------------------------------------------------------
% ELEMENTOS TEXTUAIS
% ----------------------------------------------------------
\textual

% ----------------------------------------------------------
% Introdução
% ----------------------------------------------------------
\section*{Introdução}
\addcontentsline{toc}{section}{Introdução}

O estudo da equação de transporte, também denominada equação da advecção-difusão-reação, continua sendo um ativo campo
de pesquisa, uma vez que essa equação é de fundamental importância nos problemas relacionados a aerodinâmica, meteorologia,
oceanografia, hidrologia, engenharia química e de reservatórios. Por isso, vários métodos têm sido desenvolvidos e implementados
para a solução dessa equação, que pode ser definida por:

\begin{center}
$ -k \left(\frac{\partial^2 u}{\partial x^2}+\frac{\partial^2 u}{\partial y^2}\right)
+ \beta_x(x, y) \frac{\partial u}{\partial x} + \beta_y(x, y) \frac{\partial u}{\partial y} + \gamma(x, y) u=f(x,y) $
em $\Omega = (a,b) \times (c,d)$
\end{center}

A equação de transporte é uma equação diferencial parcial, para a qual nem sempre é fácil encontrar uma solução analítica,
por isso, um método numérico pode ser útil ou essencial.

O objetivo deste trabalho é analisar como a forma de armazenamento das estruturas resultantes da discretização
da equação de transporte por diferenças finitas pode influenciar no tempo de processamento.

Para isso, foi implementado, em linguagem C, um programa que solucione a equação de transporte com condições de
contorno que serão descritas adiante. O programa aplica o algoritmo SOR em uma forma de armazenamento de matriz
pentadiagonal que será descrita e também de um modo livre de matriz.

Este relatório explanará nas futuras seções, através do referencial teórico utilizado, a implementação e os experimentos
numéricos empregados a fim de obter o melhor resultado, especialmente no quesito performance, dos algoritmos.


% ----------------------------------------------------------
% Seção de explicações
% ----------------------------------------------------------
\section{Método das Diferenças Finitas}

Esse método consiste em discretizar o domínio em que uma dada equação diferencial está sendo aplicada. Tal domínio é retangular e,
na discretização, sua extensão pela abscissa é dividida em $ n-1 $ pedações iguais e, pela ordenada, em $ m-1 $ pedaços iguais, formando
uma malha de dimensões $ n \times m $ pontos.

Depois, o método prossegue definindo a forma de aproximação, substituindo as derivadas parciais da equação por diferenças finitas.
Dessa forma, chega-se a um sistema linear onde a matriz dos coefientes é pentadiagonal. Aplica-se as condições de contorno ao sistema
linear formado, sendo que, tais condições são valores prescritos ou derivadas no contorno do domínio.

No presente trabalho, a forma de armazenamento da matriz pentadiagonal adotada foi o uso de cinco vetores, em que cada vetor representa
uma diagonal. O sistema linear então é resolvido utilizando-se o método SOR, que também é usado aqui livre de matrizes.

\section{Implementação}

\subsection{Entrada de dados}

A entrada de dados consiste de um arquivo contendo os intervalos do domínio, o número máximo de iteraçãos, a tolerância para o erro,
o fator de relaxação, as dimensões da malha de discretização e o valor prescrito no contorno.

\subsection{Armazenamento e compactação matricial}

Como o algoritmo usado aqui para a resolução do sistema linear resultante da equação de transporte é o SOR (\textit{Succesive Over Relaxation}),
onde a matriz dos coeficientes não é alterada e é pentadiagonal, o tipo abstrato de dados (TAD) desenvolvido para o armazenamento dessa matriz
consiste de cinco vetores, em que cada vetor representa uma diagonal, e um outro vetor que representa os termos independentes. As dimensões $n$
e $m$ da malha resultante da discretização também são armazenadas. Esse TAD pode ser encontrado nos arquivos \emph{coef.c} e \emph{coef.h}.
No arquivo \emph{coef.h} também se encontram macros para o armazenamento das funções $\beta_x$, $\beta_y$ e $\gamma$.

A função do TAD que dá origem ao sistema linear no programa é a \emph{criaCoef} que recebe como parâmetros os intervalos do domínio $ \Omega $.
A estrutura armazenada na TAD é então passada para funções de tratamento de contorno e para a função que implementa o algoritmo SOR, para que o
sistema linear representado seja resolvido.

\subsection{Tratamento das Condições de Contorno}

O tratamento das condições de contorno se dá no módulo do arquivo \emph{contorno.c} e de sua respectiva biblioteca \emph{contorno.h}.
Este módulo apresenta duas funções para tratamento do contorno, uma que é baseada num valor prescrito e se chama \emph{valorPrescrito}, e outra
que é baseada na derivada no contorno e se chama \emph{derivadaContorno}.

\subsection{SOR}

O módulo responsável pela solução do sistema linear através do método SOR encontra-se nos arquivos \emph{sor.c} e \emph{sor.h}.

Neste módulo, encontra-se implementado o método SOR em duas versões. A versão que recebe um TAD \emph{coef} como entrada se chama
\emph{sor} e a que é livre de matriz se chama \emph{sorLivre} e recebe como parâmetros as dimensões do domínio $\Omega$, as dimensões
da malha de discretização $n$ e $m$ e um valor prescrito para tratamento do contorno. Ambas as versões recebem um parâmetro indicando
a tolerância para o erro cometido pelo SOR, o número máximo de iteraçãos e o fator de relaxação.

% ----------------------------------------------------------
% ELEMENTOS PÓS-TEXTUAIS
% ----------------------------------------------------------
\postextual

% ---
% Título e resumo em língua estrangeira
% ---

% \twocolumn[    		% INICIO DE ARTIGO EM DUAS COLUNAS

% titulo em inglês

% ]  				% FIM DE ARTIGO EM DUAS COLUNAS
% ---

% ----------------------------------------------------------
% Referências bibliográficas
% ----------------------------------------------------------
\bibliography{abntex2-modelo-references}

% ----------------------------------------------------------
% Glossário
% ----------------------------------------------------------
%
% Há diversas soluções prontas para glossário em LaTeX. 
% Consulte o manual do abnTeX2 para obter sugestões.
%
%\glossary

% ----------------------------------------------------------
% Apêndices
% ----------------------------------------------------------

% ---
% Inicia os apêndices
% ---

% ---

% ----------------------------------------------------------
% Anexos
% ----------------------------------------------------------
\cftinserthook{toc}{AAA}
% ---
% Inicia os anexos
% ---
%\anexos


\end{document}
